\begin{enumerate}
\def\labelenumi{\arabic{enumi}.}
\item
  Escriba un PEGJS que recibe como entrada una gramática escrita en un
  lenguaje como este:

\begin{verbatim}
  a -> b 'c' | c 'b';
  b -> 'b' b | '' ;
  c -> 'c' c | '';
\end{verbatim}
\end{enumerate}

\begin{itemize}
\item
  Las variables sintácticas como \texttt{a}, \texttt{b} y \texttt{c}
  figuran en la parte izquierda,
\item
  la flecha \texttt{-\textgreater{}} indica \emph{produce} y
\item
  las alternativas de la parte derecha van separadas por barras
  \texttt{\textbar{}}.
\item
  Las partes derechas son concatenaciones de cadenas como
  \texttt{\textquotesingle{}c\textquotesingle{}},
  \texttt{\textquotesingle{}b\textquotesingle{}} o la cadena vacía
  \texttt{\textquotesingle{}\textquotesingle{}} y de variables
  sintácticas como \texttt{a}, \texttt{b} y \texttt{c}
\item
  Las reglas se terminan con un punto y coma \texttt{;}
\item
  El programa debe retornar como salida el programa PEG.js equivalente a
  la gramática de entrada. Para el ejemplo arriba, la salida sería:

\begin{verbatim}
  a = b 'c' / c 'b';
  b = 'b' b / '' ;
  c = 'c' c / '';
\end{verbatim}
\item
  Los alumnos que no hayan superado la parte práctica deben explicar
  como se compila la gramática y escribir un programa que haga uso del
  parser generado
\end{itemize}

\begin{enumerate}
\def\labelenumi{\arabic{enumi}.}
\setcounter{enumi}{1}
\itemsep1pt\parskip0pt\parsep0pt
\item
  Escriba un analizador descendente recursivo predictivo que resuelva el
  problema anterior.

  \begin{itemize}
  \itemsep1pt\parskip0pt\parsep0pt
  \item
    Los alumnos que no hayan superado la 2ª parte o la parte práctica
    deberán escribir el analizador léxico
  \item
    Si tiene pendientes las prácticas procure especialmente la precisión
    y claridad en todo el código de esta prueba.
  \end{itemize}
\end{enumerate}

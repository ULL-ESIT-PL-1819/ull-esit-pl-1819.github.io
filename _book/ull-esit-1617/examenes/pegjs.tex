\subsubsection{Ejercicios sobre PEGs}\label{ejercicios-sobre-pegs}

\begin{itemize}
\item
  ¿Que opción debemos pasar a \texttt{pegjs} para que admita que las
  acciones semánticas se especifiquen en \texttt{CoffeeScript}?
\item
  Escriba en PEGJS un traductor de expresiones aritméticas a postfijo.
  Una expresión como \texttt{a = 4*b+c} se debe traducir por
  \texttt{4 b * c + \&a =}
\item
  Escriba un PEGJS que traduzca secuencias de expresiones aritméticas
  separadas por puntos y comas a JavaScript. Una secuencia de
  expresiones como

\begin{verbatim}
b = 5;
c = 2;
a = 4*b+c;
b+c+a 
\end{verbatim}

  se debería traducir como:
\end{itemize}

\begin{verbatim}
var b = 5;
var c = 2;
var a = 4*b+c;
b+c+a
\end{verbatim}

\begin{itemize}
\item
  Escriba un peg que reconozca las sentencias \texttt{if then ...} e
  \texttt{if then ... else ...} asociando cada \texttt{else} con el
  \texttt{if} mas cercano

\begin{verbatim}
if a then if b then o else o
\end{verbatim}
\item
  Explique como se le puede pasar información al objeto Parser generado
  por PEG.js
\item
  Escriba un peg que reconozca los comentarios Pascal anidados
\item
  Escriba un peg que reconozca el lenguaje
  \[ \{ a^nb^nc^n / n \ge 1 \} \]
\item
  Indique como es la estructura de tabla de símbolos de un compilador y
  como se construye. Explique como el anidamiento de las tablas de
  símbolos refleja el ámbito de las variables
\item
  Indique como se hace el análisis de tipos en un compilador mediante la
  decoración del árbol con el atributo \texttt{tipo}. Señale las
  diferentes formas en la que se pueden resolver las cincompatibilidades
  de tipos
\end{itemize}

\paragraph{Ejercicio: Llamadas a función en
PL0.1}\label{ejercicio-llamadas-a-funciuxf3n-en-pl0.1}

En el PEGjs en
\href{https://github.com/crguezl/pegjscalc/blob/clase09052016/lib/pl0.pegjs}{pl0.pegjs}
de la rama
\href{https://github.com/crguezl/pegjscalc/tree/clase09052016}{clase09052016}
se añaden así llamadas a funciones:

\begin{Shaded}
\begin{Highlighting}[]
\NormalTok{factor = NUMBER}
       \NormalTok{/ id:ID LEFTPAR a:assign? r:(COMMA assign)* RIGHTPAR}
\end{Highlighting}
\end{Shaded}

El problema con esta regla es que acepta
\href{https://github.com/crguezl/pegjscalc/blob/clase09052016/tests/simple1err.pl0}{entradas
como esta} con una coma inicial:

\begin{Shaded}
\begin{Highlighting}[]
  \KeywordTok{while} \NormalTok{(a < }\DecValTok{10}\NormalTok{) }\KeywordTok{do} \NormalTok{a = }\FunctionTok{fact}\NormalTok{(,a}\DecValTok{+1}\NormalTok{,b);}
\end{Highlighting}
\end{Shaded}

Obviamente el problema es fácil de arreglar usando dos reglas:

\begin{Shaded}
\begin{Highlighting}[]
       \OtherTok{/ ID LEFTPAR RIGHTPAR}
\OtherTok{       /} \NormalTok{ID LEFTPAR }\FunctionTok{assign} \NormalTok{(COMMA assign)* RIGHTPAR}
\end{Highlighting}
\end{Shaded}

\textbf{¿Sabría como resolver el problema usando una sola regla?}

Puede editar el texto que sigue y pulsar \texttt{Submit} para comprobar
si su solución coincide con la del profesor. Si cree tener la solución
puede comentarla en este párrafo (signo de \texttt{+} a la derecha del
párrafo, es necesario estar registrado en GitBook).

\{\% regexp width=``100\%'', height=``100\%'', color=``\#0b3136'' ,
editorHeight=``350px'' \%\} \{\% editor \%\} factor = NUMBER / id:ID
LEFTPAR a:assign? r:(COMMA assign)* RIGHTPAR /* llamada a función*/ \{
let c = a? {[}a{]} : {[}{]}; c= c.concat(r.map(({[}\_, t{]})
=\textgreater{} t)); return \{ type: `CALL', name: id, children: c \} \}
/ ID / LEFTPAR t:assign RIGHTPAR \{ return t; \} \{\% solution \%\}
Solución aún no publicada \{\% validation \%\}
/\&\{\s*return\s+a;?\s*\}\s+r:(COMMA\s+assign)/ \{\% endregexp \%\}

\paragraph{Ejercicio: Acción Semántica para las Declaraciones de
funciones en
PL0.1}\label{ejercicio-acciuxf3n-semuxe1ntica-para-las-declaraciones-de-funciones-en-pl0.1}

En el PEGjs en
\href{https://github.com/crguezl/pegjscalc/blob/clase10052016/lib/pl0.pegjs}{pl0.pegjs}
de la rama
\href{https://github.com/crguezl/pegjscalc/tree/clase10052016}{clase09052016}
se añaden así las declaraciones de funciones:

\begin{Shaded}
\begin{Highlighting}[]
\NormalTok{functionDeclaration = FUNCTION id:ID LEFTPAR !COMMA p1:ID? r:(COMMA ID)* RIGHTPAR block SC}
                        \NormalTok{\{}
                        \NormalTok{\}}
\end{Highlighting}
\end{Shaded}

Complete el código que falta. Aplane los bloques de manera que cuando se
le de como entrada un programa como este:

\begin{Shaded}
\begin{Highlighting}[]
\KeywordTok{const} \NormalTok{A = }\DecValTok{4}\NormalTok{, }
      \NormalTok{B = }\DecValTok{30}\NormalTok{;}
\KeywordTok{var} \NormalTok{b, n;}
\KeywordTok{function} \FunctionTok{fact}\NormalTok{(n);}
  \KeywordTok{var} \NormalTok{t;}
  \KeywordTok{function} \FunctionTok{tutu}\NormalTok{(a,b,c);}
    \KeywordTok{return} \DecValTok{4}\NormalTok{;}
  \NormalTok{\{ }
    \KeywordTok{if} \NormalTok{n <= }\DecValTok{1} \NormalTok{then }\KeywordTok{return} \DecValTok{1} 
    \KeywordTok{else} \KeywordTok{return} \NormalTok{n*}\FunctionTok{fact}\NormalTok{(n}\DecValTok{-1}\NormalTok{);}
  \NormalTok{\};}
\NormalTok{\{}
  \NormalTok{n = }\DecValTok{9}\NormalTok{;}
  \NormalTok{b = }\FunctionTok{fact}\NormalTok{(n);}
\NormalTok{\}}
\end{Highlighting}
\end{Shaded}

Produzca un árbol de salida similar a este:

\begin{Shaded}
\begin{Highlighting}[]
\NormalTok{\{ }\DataTypeTok{type}\NormalTok{: }\StringTok{'BLOCK'}\NormalTok{,}
  \DataTypeTok{constants}\NormalTok{: [ [ }\StringTok{'A'}\NormalTok{, }\DecValTok{4} \NormalTok{], [ }\StringTok{'B'}\NormalTok{, }\DecValTok{30} \NormalTok{] ],}
  \DataTypeTok{variables}\NormalTok{: [ }\StringTok{'b'}\NormalTok{, }\StringTok{'n'} \NormalTok{],}
  \DataTypeTok{functions}\NormalTok{: }
   \NormalTok{[ \{ }\DataTypeTok{name}\NormalTok{: }\StringTok{'fact'}\NormalTok{,}
       \DataTypeTok{params}\NormalTok{: [ }\StringTok{'n'} \NormalTok{],}
       \DataTypeTok{type}\NormalTok{: }\StringTok{'BLOCK'}\NormalTok{,}
       \DataTypeTok{constants}\NormalTok{: [],}
       \DataTypeTok{variables}\NormalTok{: [ }\StringTok{'t'} \NormalTok{],}
       \DataTypeTok{functions}\NormalTok{: }
        \NormalTok{[ \{ }\DataTypeTok{name}\NormalTok{: }\StringTok{'tutu'}\NormalTok{,}
            \DataTypeTok{params}\NormalTok{: [ }\StringTok{'a'}\NormalTok{, }\StringTok{'b'}\NormalTok{, }\StringTok{'c'} \NormalTok{],}
            \DataTypeTok{type}\NormalTok{: }\StringTok{'BLOCK'}\NormalTok{,}
            \DataTypeTok{constants}\NormalTok{: [],}
            \DataTypeTok{variables}\NormalTok{: [],}
            \DataTypeTok{functions}\NormalTok{: [],}
            \DataTypeTok{main}\NormalTok{: \{ }\DataTypeTok{type}\NormalTok{: }\StringTok{'RETURN'}\NormalTok{, }\DataTypeTok{children}\NormalTok{: [ \{ }\DataTypeTok{type}\NormalTok{: }\StringTok{'NUM'}\NormalTok{, }\DataTypeTok{value}\NormalTok{: }\DecValTok{4} \NormalTok{\} ] \} \} ],}
       \DataTypeTok{main}\NormalTok{: }
        \NormalTok{\{ }\DataTypeTok{type}\NormalTok{: }\StringTok{'COMPOUND'}\NormalTok{,}
          \DataTypeTok{children}\NormalTok{: }
           \NormalTok{[ \{ }\DataTypeTok{type}\NormalTok{: }\StringTok{'IFELSE'}\NormalTok{,}
               \DataTypeTok{c}\NormalTok{: }
                \NormalTok{\{ }\DataTypeTok{type}\NormalTok{: }\StringTok{'<='}\NormalTok{,}
                  \DataTypeTok{left}\NormalTok{: \{ }\DataTypeTok{type}\NormalTok{: }\StringTok{'ID'}\NormalTok{, }\DataTypeTok{value}\NormalTok{: }\StringTok{'n'} \NormalTok{\},}
                  \DataTypeTok{right}\NormalTok{: \{ }\DataTypeTok{type}\NormalTok{: }\StringTok{'NUM'}\NormalTok{, }\DataTypeTok{value}\NormalTok{: }\DecValTok{1} \NormalTok{\} \},}
               \DataTypeTok{st}\NormalTok{: \{ }\DataTypeTok{type}\NormalTok{: }\StringTok{'RETURN'}\NormalTok{, }\DataTypeTok{children}\NormalTok{: [ \{ }\DataTypeTok{type}\NormalTok{: }\StringTok{'NUM'}\NormalTok{, }\DataTypeTok{value}\NormalTok{: }\DecValTok{1} \NormalTok{\} ] \},}
               \DataTypeTok{sf}\NormalTok{: }
                \NormalTok{\{ }\DataTypeTok{type}\NormalTok{: }\StringTok{'RETURN'}\NormalTok{,}
                  \DataTypeTok{children}\NormalTok{: }
                   \NormalTok{[ \{ }\DataTypeTok{type}\NormalTok{: }\StringTok{'*'}\NormalTok{,}
                       \DataTypeTok{left}\NormalTok{: \{ }\DataTypeTok{type}\NormalTok{: }\StringTok{'ID'}\NormalTok{, }\DataTypeTok{value}\NormalTok{: }\StringTok{'n'} \NormalTok{\},}
                       \DataTypeTok{right}\NormalTok{: }
                        \NormalTok{\{ }\DataTypeTok{type}\NormalTok{: }\StringTok{'CALL'}\NormalTok{,}
                          \DataTypeTok{name}\NormalTok{: }\StringTok{'fact'}\NormalTok{,}
                          \DataTypeTok{arguments}\NormalTok{: }
                           \NormalTok{[ \{ }\DataTypeTok{type}\NormalTok{: }\StringTok{'-'}\NormalTok{,}
                               \DataTypeTok{left}\NormalTok{: \{ }\DataTypeTok{type}\NormalTok{: }\StringTok{'ID'}\NormalTok{, }\DataTypeTok{value}\NormalTok{: }\StringTok{'n'} \NormalTok{\},}
                               \DataTypeTok{right}\NormalTok{: \{ }\DataTypeTok{type}\NormalTok{: }\StringTok{'NUM'}\NormalTok{, }\DataTypeTok{value}\NormalTok{: }\DecValTok{1} \NormalTok{\} \} ] \} \} ] \} \} ] \} \} ],}
  \DataTypeTok{main}\NormalTok{: }
   \NormalTok{\{ }\DataTypeTok{type}\NormalTok{: }\StringTok{'COMPOUND'}\NormalTok{,}
     \DataTypeTok{children}\NormalTok{: }
      \NormalTok{[ \{ }\DataTypeTok{type}\NormalTok{: }\StringTok{'='}\NormalTok{,}
          \DataTypeTok{left}\NormalTok{: \{ }\DataTypeTok{type}\NormalTok{: }\StringTok{'ID'}\NormalTok{, }\DataTypeTok{value}\NormalTok{: }\StringTok{'n'} \NormalTok{\},}
          \DataTypeTok{right}\NormalTok{: \{ }\DataTypeTok{type}\NormalTok{: }\StringTok{'NUM'}\NormalTok{, }\DataTypeTok{value}\NormalTok{: }\DecValTok{9} \NormalTok{\} \},}
        \NormalTok{\{ }\DataTypeTok{type}\NormalTok{: }\StringTok{'='}\NormalTok{,}
          \DataTypeTok{left}\NormalTok{: \{ }\DataTypeTok{type}\NormalTok{: }\StringTok{'ID'}\NormalTok{, }\DataTypeTok{value}\NormalTok{: }\StringTok{'b'} \NormalTok{\},}
          \DataTypeTok{right}\NormalTok{: }
           \NormalTok{\{ }\DataTypeTok{type}\NormalTok{: }\StringTok{'CALL'}\NormalTok{,}
             \DataTypeTok{name}\NormalTok{: }\StringTok{'fact'}\NormalTok{,}
             \DataTypeTok{arguments}\NormalTok{: [ \{ }\DataTypeTok{type}\NormalTok{: }\StringTok{'ID'}\NormalTok{, }\DataTypeTok{value}\NormalTok{: }\StringTok{'n'} \NormalTok{\} ] \} \} ] \} \}}
\end{Highlighting}
\end{Shaded}

Si cree tener la solución puede comentarla en este párrafo (signo de
\texttt{+} o \texttt{\#nº} a la derecha del párrafo, es necesario estar
registrado en GitBook).

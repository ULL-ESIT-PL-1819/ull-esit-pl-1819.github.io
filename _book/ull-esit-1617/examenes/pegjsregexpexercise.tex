\begin{enumerate}
\def\labelenumi{\arabic{enumi}.}
\item
  Escriba un PEGJS que recibe como entrada una expresión regular como
  \texttt{a(b.c)+\textbar{}abc} y produce como salida un árbol de
  análisis sintáctico:

\begin{verbatim}
  [ 'ALT',
    [ 'CAT',
      [ [ 'CHAR', 'a' ],
        [ '+',
          [ 'CAT', [ [ 'CHAR', 'b' ], 
                     [ 'DOT', [] ], 
                     [ 'CHAR', 'c' ] 
                   ] 
          ] 
        ] 
      ] 
    ],
    [ 'CAT', [ [ 'CHAR', 'a' ], [ 'CHAR', 'b' ], [ 'CHAR', 'c' ] ] ] 
  ]
\end{verbatim}
\end{enumerate}

En esta representación del árbol AST hemos usado un formato muy simple
mediante arrays: el primer elemento del array indica el tipo de nodo
(por ejemplo \texttt{\textquotesingle{}ALT\textquotesingle{}} para
indicar una alternativa o unión ) y el resto de elementos en el array
son los hijos de ese nodo.

No tiene por que ser esta la representación que use para su AST:
cualquier representación razonable que respete la prioridad y
asociatividad de las operaciones y facilite el recorrido del AST será
considerada correcta.

Las expresiones regulares que consideramos vienen definidas por las
siguientes reglas:

\begin{enumerate}
\def\labelenumi{\arabic{enumi}.}
\itemsep1pt\parskip0pt\parsep0pt
\item
  Si \texttt{re1}y \texttt{re2} son dos regexps entonces
  \texttt{re1 \textbar{} re2} es una regexp
\item
  La concatenacion de regexps es una regexp
\item
  Los cierres de regexps \texttt{re} como
\end{enumerate}

\begin{itemize}
\itemsep1pt\parskip0pt\parsep0pt
\item
  \texttt{re?}
\item
  \texttt{re*}
\item
  \texttt{re+} son regexps
\end{itemize}

\begin{enumerate}
\def\labelenumi{\arabic{enumi}.}
\setcounter{enumi}{3}
\itemsep1pt\parskip0pt\parsep0pt
\item
  Si \texttt{re} es una regexp entonces \texttt{(re)} entre paréntesis
  es también una regexp
\item
  El punto \texttt{\textquotesingle{}.\textquotesingle{}}, el
  \texttt{\textquotesingle{}\^{}\textquotesingle{}} y el
  \texttt{\textquotesingle{}\$\textquotesingle{}} son regexps
\item
  Los caracteres que no son metasimbolos (esto es, distintos de
  \texttt{\textquotesingle{}*\textquotesingle{}},
  \texttt{\textquotesingle{}+\textquotesingle{}},
  \texttt{\textquotesingle{}\textbar{}\textquotesingle{}}, etc.) son
  expresiones regulares
\item
  Los metasímbolos (como \texttt{\textquotesingle{}*\textquotesingle{}},
  \texttt{\textquotesingle{}+\textquotesingle{}},
  \texttt{\textquotesingle{}\textbar{}\textquotesingle{}}, etc.) cuando
  son escapados
  \texttt{\textquotesingle{}\textbackslash{}*\textquotesingle{}} se
  consideran caracteres y son también expresiones regulares
\item
  \textbf{Si no ha superado la segunda parte deberá incluir en su
  lenguaje de regexps las clases o conjuntos}:
\end{enumerate}

\begin{itemize}
\itemsep1pt\parskip0pt\parsep0pt
\item
  Una clase es una secuencia de items entre corchetes:
  \texttt{{[} items {]}} es una regexp
\item
  Los \texttt{items} pueden ser bien carácteres o bien rangos
  \texttt{char - char}
\end{itemize}
